\documentclass{report}
\usepackage[T1]{fontenc} 
\usepackage[utf8]{inputenc} 
\usepackage[backend=biber, style=ieee]{biblatex} 
\usepackage{csquotes}
\usepackage[portuguese]{babel}
\usepackage{blindtext}
\usepackage[printonlyused]{acronym}
\usepackage{hyperref}
\usepackage{graphicx}


\begin{document}
%
% Settings
%
\def\titulo{PBOX Client}
\def\data{21-04-2017}
\def\autores{Mário Liberato, Jorge Oliveira}
\def\autorescontactos{(84917) mliberato@ua.pt, (84983) jorge.am.oliveira@ua.pt}
\def\departamento{DETI}
\def\curso{MIECT}
\def\logotipo{ua.pdf}
%
%CAPA %
%
\begin{titlepage}

\begin{center}
%
\vspace*{50mm}
%
{\Huge \titulo}\\ 
%
\vspace{10mm}
%
{\Large \curso}\\
%
\vspace{10mm}
%
{\LARGE \autores}\\ 
%
\vspace{30mm}
%
\begin{figure}[h]
\center
\includegraphics{\logotipo}
\end{figure}
%
\vspace{30mm}
\end{center}
%
\end{titlepage}

% Pag Titulo %
\title{%
{\Huge\textbf{\titulo}}\\
{\Large \departamento\\ \curso}
}
%
\author{%
    \autores \\
    \autorescontactos
}
%
\date{\data}
%
\maketitle

\pagenumbering{roman}

%RESUMO%
\begin{abstract}
Resumo do trabalho..........................................................................
\end{abstract}

% Agradecimentos %
%\renewcommand{\abstractname}{Agradecimentos}
%\begin{abstract}
%\end{abstract}
%Não existem agradecimentos para este relatório

\tableofcontents
% \listoftables     
% \listoffigures    


%
\clearpage
\pagenumbering{arabic} %Numeracao fica a direita
%
\chapter{Introdução}
\label{chap.introducao} 

O tema do seguinte trabalho é a criação de um programa cliente onde é possível realizar certas operações envolvendo a interação com um servidor.
Este servidor contém um modelo de "boxes" com (ou sem) segurança onde é possível deixar mensagens de curto comprimento para consulta do seu criador. Todos os utilizadores têm possibilidade de criar uma \textbf{caixa} com um certo nome, também sendo possível existir uma chave pública ao servidor (no caso das caixas seguras). Além da criação de caixas, é possível \textbf{listar},\textbf{receber} e \textbf{enviar} mensagens para uma certa caixa.

O documento encontra-se dividido em quatro capítulos.
Sendo que no \autoref{chap.metodologia} é apresentada a metodologia seguida para a criação do cliente e
as funções do mesmo.
No \autoref{chap.resultados} são apresentados os resultados obtidos no cliente e a respetiva análise.
Finalmente, no \autoref{chap.conclusao} são apresentadas
as conclusões do trabalho.

\chapter{Metodologia}
\label{chap.metodologia}

Para a criação do cliente foi utilizada essencialmente programação em Python (para isto foram utilizadas ferramentas como IDEs, nomeadamente o PyCharm, Geany e vim) e a interface do cliente foi concebida como uma aplicação web baseada em CherryPy.

Antes da realização de qualquer programação o grupo reuniu-se e discutiu como deveria ser realizado o trabalho, que caminhos seguir até ao produto final. Foi optado ser realizada uma pequena base do cliente onde seria possível listar as caixas disponíveis no servidor, de seguida foram sendo adicionadas as funções de criar caixas, dar segurança às mesma, receber e enviar documentos às caixas. Finalmente a interface gráfica, mais apelativa ao utilizador foi introduzida. Antes de prosseguir, em cada etapa foram realizados testes para determinar erros ou falhas (em especial de segurança) para ser obtida uma experiência sem problemas ou crashes. 

\section{Descrição do cliente}
\label{subs.desc}
Nesta secção são apresentadas as funções do cliente e como foram adaptadas ao mesmo.

\subsection{Listagem}
É pretendido listar todas as caixas seguras existentes através do envio de uma mensagem \textbf{LIST}, sendo que o servidor responderá com todas caixas seguras existentes e, caso existam, as chaves públicas das mesmas. Diversas funções foram concebidas para isto, para, por exemplo, obter o nome de todas as caixas, e mostrá-los ao utilizador.


\subsection{Criação de caixas}
Esta função permite criar uma caixa através do envio da mensagem \textbf{CREATE}, a mensagem deverá conter o nome da caixa e o seu timestamp. Ainda é possível ter uma chave pública e assinatura da mensagem. Para oferecer segurança, se for optado por fornecer uma chave pública essa caixa só pode ser modificada com mensagens seguras.


\subsection{Envio de documentos para uma caixa}
Para o envio de documentos para uma caixa é necessário o envio da mensagem \textbf{PUT} contendo uma mensagem de até 65536 octetos. Se a mensagem tiver mais que este comprimento, será cortada para ter a dimensão adequada.


\subsection{Receção de um documento}
A receção de um documento de uma certa caixa é realizado através da mensagem GET contendo o seu timestamp. Foram criadas funções que permitem obter a mensagem mais antiga no servidor, assim como duas implementações de uma função que permite obter todas as mensagens existentes de uma vez. Esta função pode ainda eliminar todas as mensagens existentes ou apenas uma, sendo chamada sem utilizar o valor devolvido, ou usando as duas funções que as chamam sem retornar valores.


\subsection{Segurança}
Para segurança são utilizadas cifras assimétricas \ac{rsa} com chaves de 2048 bits que são enviadas em formato \ac{pem}. Também são necessárias Assinaturas com chaves \ac{rsa} e sínteses{sha1}, sendo isto visível na criação de caixas e obtenção dos documentos numa caixa.

\subsection{Interface Web}
Foi criada uma interface web para o cliente ser mais apelativo e de uso mais simples e familiar ao utilizador.

\chapter{Resultados e Análise}
\label{chap.resultados}

Antes demais será de notar que o grupo fez o trabalho em conjunto com sucessivos commits no git. Inicialmente o trabalho começou a usar o GitHub devido a alguns problemas com a plataforma Code.UA



\chapter{Conclusões}
\label{chap.conclusao}
Apresenta conclusões.


\chapter*{Acrónimos}
\begin{acronym}
 \acro{ua}[UA]{Universidade de Aveiro}
 \acro{deti}[DETI]{Departamento de Electrónica, Telecomunicações e Informática}
 \acro{miect}[MIECT]{Mestrado Integrado em Engenharia de Computadores e Telemática}
 \acro{rsa}[RSA]{Rivest Shamir Adleman, Iniciais dos apelidos dos fundadores deste algorítmo de criptografia}
 \acro{pem}[PEM]{Privacy-enhanced Electronic Mail}
 \acro{sha1}[SHA1]{Função hash criptográfica}
\end{acronym}


%
%\printbibliography

\end{document}
